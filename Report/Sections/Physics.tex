\section{Review of the physics}

\subsection{Speckle fields} \label{speckles}

A "speckle field" is an optical field with a random pattern of constructive and destructive interference, which gives rise to little dots of lights called "speckles". 
A speckle field can be studied statistically, usually only considering first- and second-order statistics, i.e. the intensity distribution and the two-point 
correlation function. \\

The intensity correlation function of a speckle field with intensity $I(\bm x) \propto \langle |\bm E(\bm x)|^2 \rangle$, and its 
normalized version, are given by (assuming that they only depend on the separation between the two points)

\begin{equation}
    G(\bm s) = \langle I(\bm x)I(\bm x + \bm s) \rangle, \qquad g(\bm s) = \frac{G(\bm s)}{\langle I(\bm x)^2 \rangle}.
\end{equation}

$\langle \ \rangle$ denotes the average over an ensemble of fields. The intensity correlation function can be computed directly from the measured intensity. 
It characterizes the average size of the speckles, but it doesn't contain any information about the phase relation between two points. \\

By contrast, the field correlation function

\begin{equation}
    \Gamma(\bm s) = \langle \bm E(\bm x)^* \cdot \bm E(\bm x + \bm s)\rangle, \qquad \gamma(\bm s) = \frac{\Gamma(\bm s)}{\langle |\bm E(\bm x)|^2 \rangle}
\end{equation}

also contains the information about the phase relation, but it cannot be computed from a direct measurement of the field. Instead, it is necessary to use 
interferometry to obtain the necessary information about the average phase relation. 

\subsection{Measuring the field correlation function} \label{meas_corr}

Consider a double slit, neglecting for now the finite slit width effects, probing a $1$D speckle field. The far-field interference pattern is, for a slit 
separation $s$,

\begin{equation} \label{int_field}
    E_{\text{screen}}(y) = E\left(-\frac s2\right)\frac{e^{ikr_1}}{r_1} + E\left(\frac s2\right)\frac{e^{ikr_2}}{r_2},
\end{equation}

where $k = 2\pi/\lambda$ and $r_{1, 2}$ is the distance of the point on the screen $y$ from the two slits. The approximation to first order
$r_2 - r_1 = \sqrt{z^2 + (y - s/2)^2} - \sqrt{z^2 + (y + s/2)^2} \approx ys/2z$ is valid in the far-field regime ($z$ is the longitudinal propagation distance). 
Going over to the averaged intensity, 

\begin{equation}
    \langle |E_{\text{screen}}(y)|^2 \rangle = \frac 2{z^2}\Gamma(0) + \frac 2{z^2} \Re \left\{ \left\langle E\left(-\frac s2\right)^*E\left(\frac s2\right) \right\rangle  \exp\left( \frac{2\pi}{\lambda z} ys \right) \right\},
\end{equation}

where $\Gamma(0) = \langle |E(-s/2)|^2 \rangle =  \langle |E(s/2)|^2 \rangle$ is the average intensity of the speckle field. In the expression there appears the field correlation 
function $\Gamma(s) = |\Gamma(s)|e^{i\phi(s)}$. Therefore, modulo a $z^2$ factor,

\begin{equation} \label{part_cor_patt}
    \langle |E_{\text{screen}}(y)|^2 \rangle = 2 \, \Gamma(0) + 2 \, |\Gamma(s)| \cos\left[ \frac{2\pi}{\lambda z} sy + \phi(s) \right].
\end{equation}

The maximum intensity $I_{\text{max}}$ is attained when the cosine is $+1$, the minimum $I_{\text{min}}$ when the cosine is $-1$. Therefore 

\begin{equation}\label{vis_formula}
    \mathcal V = \frac{I_{\text{max}} - I_{\text{min}}}{I_{\text{max}} + I_{\text{min}}} = \frac{|\Gamma(s)|}{\Gamma(0)} = |\gamma(s)|.
\end{equation}

Finally, from the expression \eqref{part_cor_patt} it can be seen that the phase $\phi(s)$ of the correlation function determines the value of the pattern when 
$y = 0$. Thus, if $\langle |E_{\text{screen}}(0)|^2 \rangle = I_{\text{min}}$, then $\phi(s) = \pi$ and $e^{i\phi(s)} = -1$; if 
$\langle |E_{\text{screen}}(0)|^2 \rangle = I_{\text{max}}$, then $\phi(s) = 0$ and $e^{i\phi(s)} = +1$.

\subsection{Finite slit width effects}

In practice, the slits have a finite width (in the simulation it is of $a = 1 \, \mathrm{mm}$), and therefore the interference pattern is modulated by a certain 
profile. In the case of coherent light, the profile is a $\mathrm{sinc}$ function (the diffraction pattern from a single slit), but in the case of partially 
coherent light the profile is more complicated. The equation \eqref{int_field} becomes, taking into accout the finite width,

\begin{equation}
    E_{\text{screen}}(y) = \int_{-s/2-a/2}^{-s/2 + a/2} \mathrm dx \, E(x)\frac{e^{ikr(x)}}{r(x)} + \int_{s/2-a/2}^{s/2 + a/2} \mathrm dx \, E(x)\frac{e^{ikr(x)}}{r(x)},
\end{equation}

where $r(x) = \sqrt{z^2 + (y - x)^2}$. Without going over the whole calculation, the resulting pattern is, modulo a $z^2$ factor, and 
neglecting the partial coherence of the field over a single slit,

\begin{equation} \label{theopatt}
    \langle |E_{\text{screen}}(y)|^2 \rangle = 2a^2 \Gamma(0) \, \mathrm{sinc}  \left( \frac{\pi a y}{\lambda z} \right)^2 \left[ 1 + |\gamma(s)| \cos\left( \frac{2\pi}{\lambda z}sy +\phi(s) \right) \right]
\end{equation}

Actually, there are also the effects due to the partial coherence of the field over the span of a single slit; in the code, I accounted for those effects by 
inserting a corrective multiplicative constant in the amplitude (this correction is only empirically justified).

\subsection{Computing the field correlation function} \label{comp_corr}

The spatial filtering of the speckle field determines its second order statistics by the Wiener-Khintchine theorem, which states that the two-point correlation 
function is the fourier transform of the spectral density (which is proportional to the field in the focal plane of the spatial filter). Therefore, if $b$ denotes 
the width of the filtering slit, and assuming an initially $\delta$-correlated field (so that its spectral density is a constant), it is easy to obtain eqn. 
\eqref{rect-corr}. \\

The software will also allow the option of a gaussian filtering. In that case, the field correlation function in the double slit plane would be 

\begin{equation}
    \gamma(s) = \exp\left[-2\left(\frac{\pi bs}{\lambda f}\right)^2\right].
\end{equation}

